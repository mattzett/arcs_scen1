\documentclass[11pt,letterpaper]{article}

\usepackage[pdftex]{graphicx}
\usepackage{natbib}
\usepackage{fullpage}
\usepackage{lineno}
\usepackage{multirow}
\usepackage{wrapfig}
\usepackage{amsmath}
\usepackage{amssymb}
\usepackage{sidecap}
\usepackage{hyperref}

\begin{document}

\setlength{\parindent}{0mm}
\setlength{\parskip}{0.4cm}

\bibliographystyle{apalike}

%\modulolinenumbers[5]
%\linenumbers

\title{3D Elliptic Equations for GEMINI}

\maketitle

\tableofcontents

\pagebreak


This document describes the mathematical formulation of the ARCS scenario 1 data inversion problem.  The measurements being made for scenario 1 will result in production of maps of magnetic field fluctuations $\Delta \mathbf{B}$, and flows $\mathbf{v}$.  These will be converted, as part of standard data processing, into parallel current density and electric field:
\begin{equation}
\mathbf{J} = \nabla \times \left( \frac{\Delta \mathbf{B}}{\mu_0} \right)
\end{equation}
\begin{equation}
\mathbf{E} = -\mathbf{v} \times \mathbf{B}
\end{equation}


\section{Physical constraints}

\subsection{Parametric approximations of ionospheric conductivities}

\subsection{Ionospheric potential problems sans source terms}

\end{document}