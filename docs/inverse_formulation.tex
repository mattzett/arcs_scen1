\documentclass[11pt,letterpaper]{article}

\usepackage[pdftex]{graphicx}
\usepackage{natbib}
\usepackage{fullpage}
\usepackage{lineno}
\usepackage{multirow}
\usepackage{wrapfig}
\usepackage{amsmath}
\usepackage{amssymb}
\usepackage{sidecap}
\usepackage{hyperref}

\begin{document}

\setlength{\parindent}{0mm}
\setlength{\parskip}{0.4cm}

\bibliographystyle{apalike}

%\modulolinenumbers[5]
%\linenumbers

\title{ARCS scenario 1 data inversions}

\maketitle

\tableofcontents

\pagebreak


This document describes the mathematical formulation of the ARCS scenario 1 data inversion problem.  


\section{Summary}

Measurements being made for ARCS scenario 1 crossings will result in production of maps of magnetic field fluctuations $\delta \mathbf{B}$, and flows $\mathbf{v}$ as described in the various instrument and data processing section.  These will be converted, as part of standard data processing, into parallel current density and electric field:
\begin{equation}
\mathbf{J} = \nabla \times \left( \frac{\delta \mathbf{B}}{\mu_0} \right); \qquad \mathbf{E} = -\mathbf{v} \times \mathbf{B}
\end{equation}
Because this is a swarm of satellites we will have \emph{datamaps} (2D images) of these parameters; additionally we note that the measurements may be directly used to also produce datamaps of Poynting flux:
\begin{equation}
\mathbf{S} = \mathbf{E} \times \frac{\delta \mathbf{B}}{\mu_0}
\end{equation}
In the case of scenario 1 data,  key unknown physical parameters are datamaps of ionospheric Pedersen and Hall conductances - without these we do not have a full picture of electrodynamics and energy flow in the auroral system and cannot, then, fully unlock the scientific potential of the large amount of scenario 1 data.  

Scenario 1 current density and Poynting flux Datamaps are related to each other through electromagnetic conservation laws:  current continuity and the Poynting theorem.  Under assumptions of steady state conditions and equipotential field lines (appropriate at scales relevant to our science questions), these can be reduced to:
\begin{eqnarray}
J_\parallel &=& \Sigma_P \nabla \cdot \mathbf{E}_\perp + \nabla \Sigma_P \cdot \mathbf{E}_\perp - \nabla \Sigma_H \cdot \left( \mathbf{E}_\perp \times \hat{\mathbf{b}} \right) \\
S_{\parallel}  &=& - \Sigma_P E^2
\end{eqnarray}
with unknown Pedersen and Hall conductances.  These conductances may, in principle, be solved by simply inverting this system of equations.  In practice there are many different methods that could work to perform this inversion.  A first glance the system appears even-determined on account of the fact that we have two unknown fields (conductances) and two input datamaps (conserved variables current density and Poynting flux).  It is worth noting however, that the component of the Hall conductance gradient along the electric field direction is explicitly not part of this equation and so is in the null space of the problem.  Thus in addition to conservation laws additional prior information is needed.  For ARCS data processing this will come in two forms (both of which may not always be necessary depending on noise conditions):  (1) regularization that places constrains on solution norm or smoothness (i.e. Tikhonov regularization), and/or (2) inclusion of model information that further correlates Pedersen conductance (which is well-constrained by laws above) to Hall conductance (which requires further constraints).   This could come in the form of model-based inversions (e.g. using GEMINI-GLOW) or simply parameerizations based, e.g. on the Robinson formulas or updated version of these formulas.  



\section{Physical constraints:  simplification of general conservation laws}

Conservation of charge in electromagnetic system is described by the current continuity equation:
\begin{equation}
  \frac{\partial \rho}{\partial t} + \nabla \cdot \mathbf{J} = 0
\end{equation}
In a steady state this reduces to:
\begin{equation}
\nabla \cdot \mathbf{J} = \frac{\partial J_\parallel}{\partial z} +  \nabla_\perp \cdot \mathbf{J}_\perp = 0,
\end{equation}
where the $z-$ direction represent altitude in a locally Cartesian coordinate system.  Integrating with respect to altitude:
\begin{equation}
\int \frac{\partial J_\parallel}{\partial z} dz +  \int \nabla_\perp \cdot \mathbf{J}_\perp dz = J_\parallel(\max(z)) - J_\parallel(\min(z)) + \nabla_\perp \cdot \left(  \Sigma \cdot \mathbf{E}_\perp \right) = 0
\end{equation}
Which can be expanded out and solved for the parallel current at the top of the domain, if the bottom current is assumed to be zero:
\begin{equation}
J_\parallel = - \nabla_\perp \cdot \left(  \Sigma \cdot \mathbf{E}_\perp \right) =  - \Sigma_P \nabla \cdot \mathbf{E}_\perp - \nabla \Sigma_P \cdot \mathbf{E}_\perp + \nabla \Sigma_H \cdot \left( \mathbf{E}_\perp \times \hat{\mathbf{b}} \right)
\end{equation}
The current continuity equation to be used in the ARCS analysis is then:
\begin{equation}
\boxed{
J_\parallel = -\Sigma_P \nabla \cdot \mathbf{E}_\perp - \nabla \Sigma_P \cdot \mathbf{E}_\perp + \nabla \Sigma_H \cdot \left( \mathbf{E}_\perp \times \hat{\mathbf{b}} \right)
} \label{eqn:continuity}
\end{equation}
Note that this equation effectively has two unknown fields $\Sigma_P,\Sigma_H$, but represents only one physical constraint; hence additional information is needed.  This is provided by conservation of electromagnetic energy, viz. the Poynting theorem:
\begin{equation}
\frac{\partial w}{\partial t} + \nabla \cdot \mathbf{S} = - \mathbf{J} \cdot \mathbf{E}
\end{equation}
Similar to the assumptions made to produce Equation \ref{eqn:continuity} we neglect time-dependent terms and proceed to integrate the equation along a geomagnetic field line:
\begin{equation}
S_{\parallel,top} - S_{\parallel,bottom} + \nabla_\perp \cdot \mathbf{\mathcal{S}}_\perp = - \Sigma_P E^2
\end{equation}
where $\mathbf{\mathcal{S}}_\perp$ is the column integrated perpendicular Poynting flux.  If we further assume that there is no Poynting flux through the bottom of the ionosphere or the lateral sides of our volume of interest (i.e. net incoming D.C. Poynting flux is dissipated) a simple relation between parallel Poynting flux and Pedersen conductance.  
\begin{equation}
\boxed{
S_{\parallel}  = - \Sigma_P E^2
} \label{eqn:poynting}
\end{equation}


\section{Estimating conductances}

Several different procedures can be developed for converting the maps of electric field and Poynting flux into conductances. Two approach are discussed here.   

Equation \ref{eqn:poynting} fully specifies the Pedersen conductance given quantities that are measurable by scenario 1 experiments, so the most obvious path would be to then provide the Pedersen conductance to Equation \ref{eqn:continuity}.  Superficially, the equation allows solution for the gradient of the Hall conductance and in principle one would need to compute a line integral of this quantity to solve for Hall conductance:
\begin{equation}
\Sigma_H(\mathbf{r}_2)-\Sigma_H(\mathbf{r}_1) = \int_{\mathbf{r}_1}^{\mathbf{r}_2} \nabla \Sigma_H \cdot d \mathbf{r}
\end{equation}
Moreover, one would also need the value of the Hall condutance at some reference point $\mathbf{r}_1$ to complete the solution for Hall conductance.  While it may be possible to choose a point with low density and assume zero Hall conductance at that reference point there is a more serious issue with this approach and with the set of physical constraints being used, more generally.  Equation \ref{eqn:continuity} only provides constraints on the derivative of the Hall conductance \emph{in the direction of the $\mathbf{E} \times \mathbf{B}$ drift}.  Thus, there is information about the Hall conductance (namely the variation in the direction of the electric field) that is completely unconstrained by current continuity.  As a result, the Hall conductance lies partly in the null space of the problem defined by Equations \ref{eqn:continuity} and \ref{eqn:poynting} and some additional assumptions/information/regularization will be required to solve the inverse problem.  

Another approach to the inverse problem would be to view the conservation laws as constraints to be combined together with other prior information in the form of, e.g., smoothness constraints.  Here we rewrite the physical constraints in a matrix form to facilitate application of results from linear inverse theory.  Field quantities can be ``flattened'' into vectors using column major ordering and then operators can be represented through matrix operations.  The latter step can be understood as a decomposition of the derivative operations into finite difference matrices:
\begin{equation}
\underline{j} = - \underline{\underline{I}} ~ \underline{p} \left( \nabla \cdot \mathbf{E}_\perp \right) - \underline{\underline{L}}_x \underline{p} E_x - \underline{\underline{L}}_y \underline{p} E_y + \underline{\underline{L}}_{E \times B} \underline{h} E_\perp
\end{equation}
\begin{equation}
\underline{s} = - E^2 \underline{\underline{I}} ~ \underline{p}
\end{equation}
Concatenating the unknown conductances into a single vector we get:
\begin{equation}
\underline{x} \equiv \left[ \begin{array}{c} \underline{p} \\ \underline{h} \end{array} \right]
\end{equation}
The left-hand sides of each conservation law (i.e. measurements) are similarly stacked:
\begin{equation}
\underline{b} \equiv \left[ \begin{array}{c} \underline{j} \\ \underline{s} \end{array} \right]
\end{equation}
Finally the right-hand side operations may be expressed in block diagonal form:
\begin{equation}
\underline{\underline{A}} \equiv \left[ \begin{array}{cc} -\underline{\underline{I}}  \left( \nabla \cdot \mathbf{E}_\perp \right) -  \underline{\underline{L}}_x  E_x - \underline{\underline{L}}_y E_y  & ~ \underline{\underline{L}}_{E \times B} E_\perp \\ -E^2 \underline{\underline{I}} & \underline{\underline{0}} \end{array} \right]
\end{equation}
Yielding our full set of constrains as:
\begin{equation}
\underline{\underline{A}} ~ \underline{x} = \underline{b}
\end{equation}
As discussed previously this system will not be full-rank, but serves as a starting point for a suitable generalized inverse for this problem.  As a final note the full system has size $2 \cdot N \cdot M \times 2 \cdot N \cdot M$; where $N,M$ are the $x,y$ size of the data maps provided by instrument teams.  


\section{Maximum likelihood estimators}

The maximum likelihood estimator, assuming Gaussian-distributed noise is (note we drop the underline notation here for brevity):
\begin{equation}
\hat{x}_{ML} = \left( A^T A  \right)^{-1} A^T b
\end{equation}
The matrix to be inverted here is singular for reasons noted previously; we adopt a Tikhonov regularization scheme to mitigate this:
\begin{equation}
\hat{x} = \left( A^T A  + \lambda I \right)^{-1} A^T b
\end{equation}
where $\lambda$ is a regularization parameter.  This approach regularizes the norm of the solution and coerces it to favor small norms.  One could also enforce any other conditions that can be expressed as a linear operation, yielding:
\begin{equation}
\hat{x} = \left( A^T A  + \lambda  \Gamma^T \Gamma \right)^{-1} A^T b
\end{equation}
where $\Gamma$ is an operator describing smoothness (e.g. Laplacian) or variation (gradient).  We find that the laplacian works well to keep the reconstructions as smooth as possible.  

We can add in an offset term, i.e. solve a problem of the form:
\begin{equation}
\hat{x} = \min_x \left\{  || Ax -b ||^2 + || \Gamma x||^2 +  || x - x_0 ||^2 \right\}
\end{equation}
Which solves the least squares problem subject to constraints on smoothness and some expected value for the solution.  The solution is then given by:
\begin{equation}
\hat{x} = \left( A^T A  + \lambda  \Gamma^T \Gamma \right)^{-1} \left( A^T b - \Gamma^T \Gamma x_0\right)
\end{equation}
In the case of the problem of estimating the ionospheric conductances the Hall conductance could constrained to not vary too far from the Pedersen conductivity.  

Lastly it may be advantageous to recast the current continuity in terms of the ratio of Hall to Pedersen conductance.  This retains the linearity of the problem only if the Pedersen conductance is known \emph{a priori}.  
\begin{equation}
\nabla \left(  \frac{\Sigma_H}{\Sigma_P}  \right) \cdot \mathbf{E} \times \hat{\mathbf{b}} +  \left(  \frac{\Sigma_H}{\Sigma_P}  \right) \frac{\nabla \Sigma_P}{\Sigma_P}  \cdot \mathbf{E} \times \hat{\mathbf{b}} = \frac{J_\parallel}{\Sigma_P} + \nabla_\perp \cdot \mathbf{E}_\perp + \frac{\nabla \Sigma_P}{\Sigma_P}  \cdot  \mathbf{E}_\perp
\end{equation}
This problem can be expressed in matrix form, similar to the approaches described above, and also solve via regularized inverses.  Doing so in a linear fashion does require one to first solve for the Pedersen conductance using the Poynting theorem.  Such a formulation has the benefit that you can regularize deviations from a set conductance ratio.   If a ratio of 1 is used this is equivalent to assuming an average energy of 2.5 keV for the precipitating particles.  



\section{Error Covariance}


\section{Connections to precipitating electrons}

Conductances are ultimately driven by electron precipitation, here encapsulated in terms of total energy flux $Q$ and average energy $E_{av}$ - these precipitation parameters are what is needed to ultimately drive the GEMINI simulations.  

One of the simplest parameterizations of conductance is the Robinson formulas:
\begin{equation}
\Sigma_P = \frac{40 E_{av}}{16+E_{av}^2} \sqrt{Q}
\end{equation}
\begin{equation}
\frac{\Sigma_H}{\Sigma_P} = 0.45 E_{av}^{0.85}
\end{equation}
Using these to constrain conductances creates a physical correlations between them that does not exist using just the constraints from conservation laws.  



\end{document}