\documentclass[11pt,letterpaper]{article}

\usepackage[pdftex]{graphicx}
\usepackage{natbib}
\usepackage{fullpage}
\usepackage{lineno}
\usepackage{multirow}
\usepackage{wrapfig}
\usepackage{amsmath}
\usepackage{amssymb}
\usepackage{sidecap}
\usepackage{hyperref}

\begin{document}

\setlength{\parindent}{0mm}
\setlength{\parskip}{0.4cm}

\bibliographystyle{apalike}

%\modulolinenumbers[5]
%\linenumbers

\title{3D Elliptic Equations for GEMINI}

\maketitle

\tableofcontents

\pagebreak


This document describes the mathematical formulation of the ARCS scenario 1 data inversion problem.  The measurements being made for scenario 1 will result in production of maps of magnetic field fluctuations $\Delta \mathbf{B}$, and flows $\mathbf{v}$.  These will be converted, as part of standard data processing, into parallel current density and electric field:
\begin{equation}
\mathbf{J} = \nabla \times \left( \frac{\Delta \mathbf{B}}{\mu_0} \right)
\end{equation}
\begin{equation}
\mathbf{E} = -\mathbf{v} \times \mathbf{B}
\end{equation}
In the case of scenario 1 data, the key unknown physical parameters are the ionospheric Pedersen and Hall conductances.  

\section{Physical constraints}

The current continuity equation, here used in field-integrated form, relates electric fields and conductances to current:
\begin{equation}
\boxed{
J_\parallel = \Sigma_P \nabla \cdot \mathbf{E}_\perp + \nabla \Sigma_P \cdot \mathbf{E}_\perp - \nabla \Sigma_H \cdot \left( \mathbf{E}_\perp \times \hat{\mathbf{b}} \right)
} \label{eqn:continuity}
\end{equation}
Note that this equation effectively has two unknown fields $\Sigma_P,\Sigma_H$, but represents only one physical constraint; hence additional information is needed.  This is found in the form of the Poynting theorem:
\begin{equation}
\frac{\partial w}{\partial t} + \nabla \cdot \mathbf{S} = - \mathbf{J} \cdot \mathbf{E}
\end{equation}
Similar to the assumptions made to produce Equation \ref{eqn:continuity} we neglect time-dependent terms and proceed to integrate the equation along a geomagnetic field line:
\begin{equation}
S_{\parallel,top} - S_{\parallel,bottom} + \nabla_\perp \cdot \mathbf{\mathcal{S}}_\perp = - \Sigma_P E^2
\end{equation}
where $\mathbf{\mathcal{S}}_\perp$ is the column integrated perpendicular Poynting flux.  If we further assume that there is no Poynting flux through the bottom of the ionosphere or the lateral sides of our volume of interest (i.e. net incoming D.C. Poynting flux is dissipated) a simple relation between parallel Poynting flux and Pedersen conductance.  
\begin{equation}
\boxed{
S_{\parallel}  = - \Sigma_P E^2
}
\end{equation}

\end{document}